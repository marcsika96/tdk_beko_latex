\chapter{Összefoglaló és jövőbeli munkák}

A dolgozatomban sikerrel megvalósítottam egy többlépéses gráflekérdezés generálási folyamatot, amely képes volt Neo4j adatbázis által beolvasható és feldolgozható lekérdezéseket előállítására. A generátor paraméterezhető a vizsgálni kívánt adatbázis tartalmával (címkekészlet), valamint a lekérdezések mennyiségével és bonyolultságával.
Ezen felül biztosítja lekérdezések diverzitását, így minden kiadott lekérdezés különbözik a többitől.
Az általam készített keretrendszert teljesítmény és diverzitás szempontjából is kiértékeltem egy ipari esettanulmányon, valamint korrelációt is találtunk a lekérdezések bonyolultsága és futásideje között.
Továbbá kiemelendő, hogy dolgozatomban számos technológiát alkalmaztam:
 \begin{itemize}
 	\item Gráfadatbázisok területén: Cypher gráfadatbázisnyelv, Neo4j adatbázis, Train Benchmark esettanulmány.
 	\item Modellezés területén: EMF modellező keretrendszer, Xtext technológia lekérdezések beolvasáshoz és sorosításhoz (konkrétan a slizaa nyelvtant használva), \textsc{Viatra} gráfmintaillesztő rendszer jólformáltsági kényszerek meghatározására, és Xtend-et modellek utófeldolgozására.
 	\item Matematikai eszközök:\textsc{Viatra} Solver gráfgeneráláshoz, illetve a prototipizáláshoz Alloy gráfgenerátor Sat4j mögöttes SAT megoldóval (amely skálázódási okok miatt nem volt alkalmas komolyabb mérések elvégzéséhez).
 	Elméleti eredményként elmondható, hogy sikerrel alkalmaztam egy fejlett gráfgenerálási algoritmust
 \end{itemize}

Elméleti eredményként elmondható, hogy Cypher nyelv slizaa nyelvtanának nyelvtani szabályait absztrakt szintaxis gráfon értelmezhető gráfmintákként formalizáltam \textsc{Viatra} nyelven. Ezáltal a nyelv feldolgozhatóvá vált logikai következtetőkkel, amit sikerrel alkalmaztam lekérdezések szintetizálásához. Ezen felül a Cypher nyelv szimmetrikus megfogalmazásait modell-szimmetriaként fogalmaztam meg, így javítva a lekérdezések diverzitását.

Sikerként mondható el, hogy már sikerült is találni szemantikus eltérést a tanszékemen fejlesztett InGraph 
\cite{MartonSB17} és a referenciaimplementáció Neo4j rendszerekben. Ezen kívül felkerestük a Neo4j fejlesztőit is, akiknek felkeltette az érdeklődését a fejlesztett eszköz.

Jövőbeli munkaként az általam készített lekérdezés generátor kiegészíthető lehet gráfadatbázis-tartalmak előállításával is. Az így előálló rendszer akár egy komplett környezetet biztosíthatna gráfadatbázisok tesztelésére és teljesítménymérésére, előállítva a teljes bemenetet.
