\chapter{Gráflekérdezések automatikus generálása}
\section{Lekérdezések generálása}

\subsection{Nyelv minimalizásása}
\subsection{Jólformáltsági kényszerek betartása}
Ahhoz, hogy a nyelvtan alapján olyan modelleket tudjunk generálni, amelyeket vissza tudunk majd fordítani szintaktikailag helyes lekérdezésekre, ahhoz kényszereket kell felállítanunk, amelyek betartására kötelezzük a modellgenerátort.


Én a következő kényszereket írtam fel:
\viatraStyle{pattern x(s:State)  {State(s);}}, ahol a \viatraStyle{s} változó jelöli a\ldots

\begin{lstlisting}[style=viatrasmall]
pattern goodReferenceToVariable(q : SinglePartQuery, match : Match, vari : VariableDeclaration,  variRef : VariableRef ){	
	SinglePartQuery.readingClauses(q ,match);
	Match.^pattern.patterns.^var(match, vari);
	SinglePartQuery.^return.body.returnItems.items.expression(q , variRef);
	VariableRef.variableRef(variRef, vari);
}	

@Constraint(severity = "error", key ={variRef}, message ="error")
pattern notGoodReferenceToVari(variRef : VariableRef){
	neg find goodReferenceToVariable(_, _, vari ,variRef );
	VariableRef.variableRef(variRef,vari);
}

\end{lstlisting}


A fenti két kényszer azt biztosítja, hogy a lekérdezésnek, amit generálunk mindenképpen meglegyen a két fő része. A match részben határozzuk meg azt a bizonyos mintát, amit keresünk, a return részben pedig visszatérünk az erre a mintára illeszkedő gráfrészletekkel.

Ezen belül ki kell kössük szintén, hogy a két fő rész ne csak csonkként vagy helytelenül generálódjon, hanem az egész elengedhetetlen részfa megszülessen.

Ezért van szükség a return ágon az alábbi kényszerekre:



\subsection{Diverzitás biztosítása}


\section{Automatizálás}


