\chapter{\evaluation}

Ebben a fejezetben kiértékelem munkámat mialatt megválaszolom a következő kérdéseket:

\begin{itemize}
	\item \textbf{Kérdés1}: Hogy aránylik egymáshoz az előfeldolgozás, a generálás és az utófeldolgozás időtartama?
	\item \textbf{Kérdés2}: Hogy skálázódik a generálás modell méret szempontjából?
	\item \textbf{Kérdés3}: Hogy skálázódik a generálás modell darabszám szempontjából?
	\item \textbf{Kérdés4}: Mennyire diverzek a lekérdezések? (egymás utáni 50 illetve 50 független) 
\end{itemize}


K2 : A : (12x, 50db, 20 db elem), A+(12x, 50db, 10 db elem), A-(12x, 50db, 30 db elem)
K3 : B(12x, 1db, 5-10-15-...50), B+(12x, 10, 5-10-15-...50) , B++=(12x, 30, 5-10-15-...50) 

\section{Mérési környezet felállítása}


A méréseket eclipse fejlesztői környezetben végeztem. Ahhoz, hogy bemelegítsem a modell generátort memóriakezelés és optimalizálás szepmontjából  
 To account
for warm-up effects and memory handling of the Java 8 virtual
machine, we added an extra 20 runs before the actual measurements



\section{Benchmark eredmények}
\section{Feature fedés eredmények} 

\section{Diverzitás mérése}