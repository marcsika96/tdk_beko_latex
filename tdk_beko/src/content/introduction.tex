%----------------------------------------------------------------------------
\chapter{\bevezetes}
%----------------------------------------------------------------------------

Napjainkban az adatokat többféle formátumban is tárolják. Ezek közé tartoznak a gráfadatbázisok, ahol  csomópontok reprezentálják az entitásokat és az élek az entitások közötti kapcsolatokat. Az adatstruktúrához illeszkedve többféle gráflekérdező nyelv jött létre, amelyek képesek komplex struktúrák felírására.

A gráfmintaillesztő rendszerek tesztelése azonban komoly kihívást jelent, főképp automatizált megoldásokban nem bővelkedünk. A legnagyobb kihívást ebben az esetben a változatos modellek és lekérdezések automatikus és szisztematikus előállítása jelenti, melyek tesztbemenetként szolgálnak. Továbbá, gráfadatbázisok teljesítménymérését is nagyban segítené  az automatikusan előállított modellkészlet.

Dolgozatom célja hogy ezekre a problémákra megoldást találjak. Kutatásom során   megmutatom, hogy egy automatikusan előállított diverz modell halmazzal, amelynek modelljei lekérdezésként értelmezhetőek egy gráfmintaillesztő rendszerben (pl.: VIATRA vagy Neo4j), hogyan lehetséges  az adott gráfmintaillesztő rendszer tesztelése.

Munkám során fejlett logikai következtetők alkalmazásával állítok elő modelleket, melyek diverzitását szomszédsági formákkal (neighborhood shape-ek) biztosítom. A logikai következtetők eredményeit lekérdezésekként, és adatbázisok tartalmaként egyaránt értelmezhetjük, amelyek eredményei különböző megvalósításokkal összehasonlíthatóvá válnak. A megoldásomat egy esettanulmány keretében prezentálom.

Ezzel a módszerrel lehetővé válik, nagyobb megbízhatóságú gráfmintaillesztő rendszerek fejlesztése olcsóbban. Illetve egy ekkora modell halmaz különböző gráfmintaillesztő rendszerekben lekérdezésekre fordítva és a válaszidőket lemérve teljesítménymérésekre is használható.




\textbf{Kontextus.} Napjainkban ....

\textbf{Problémafelvetés.} Azonban ...

\textbf{Célkitűzés.} Dolgozatom célja ...

\textbf{Kontribúció.} Dolgozatomban bemutatok ...

\textbf{Hozzáadott érték.} Ezáltal ...

\textbf{Dolgozat felépítése.}A második fejezetben bemutatom a dolgozat megértéséhez szükséges háttérismereteket. blabla ...

