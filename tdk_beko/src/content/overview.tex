\chapter{\attekintes}

\section{Funkcionális áttekintés}

Munkám célja hogy mutasson egy olyan koncepciót, amelynek segítségével lehetséges gráfmintaillesztő
 rendszerek tesztelése több aspektusból. Ezt úgy viszem véghez, hogy a rendszer nyelvén generálok
 mintákat/lekérdezéseket, amelyeket aztán a lekérdező motoron futtatok. Mivel generálás során nem 
 akadály az hogy sok és diverz lekérdezést készítsek el ezért a nagy lekérdezéshalmaz alkalmassá válik
 arra hogy teljesítményben tesztelje a lekérdező rendszereket. Illetve ha egyes lekérdezéseken nem tud
 futtatni a rendszer akkor kiderül az is hogy milyen funkcionalitásokat nem fed még le. Illetve egy 
 referencia implementáció segítségével azt is ellenőrizni tudjuk, hogy a megfelelő válaszolakt adja-e
 a rendszer, hibás-e a működése. A koncepciómat egy Neo4J \cite{neo4j} gráf adatbázison mutatom be 
 miközben a lekérdezéseket az ehhez kifejlesztet Cypher \cite{Cypher} nyelven generálom. 
 
A generálást gráfok segítségével végzem. Ezért van szükségem a nyelv szintaktikájának metamodelljére,
aminek alapján példánymodelleket tudok generálni az adott nyelv felépítését betartva. Azonban vannak 
olyan szabályok amelyeket a metamodell nem fejez ki, betartásuk nélkül viszont a generált példánygráfok
nem értelmezhetőek Cypher nyelvű lekérdezésekként. Ezen szabályok betartására jólformáltsági kényszereket 
írok fel, amelyeket a generátor támpontként használ a lekérdezések készítése során. Ahhoz hogy mindezt 
véghez tudjam vinni választanom kellett egy gráf generáló szoftvert ami kényszerek és egy metamodell
alapján képes hatékonyan példánymodellek generállására. Ehhez a ViatraSolver nevű alkalmazást választottam,
ami eclipse fejlesztői környezetben működik, a jólformálsági kényszereket pedig Viatra nyelven írtam fel.

Miután a példánygráfok létrejönnek már csak le kell őket fordítani Cypher nyelvű lekérdezésekre.
A fordítás során ki kell tölteni a gráfgenerátor által üresen hagyott értékeket. Illetve át parszolni a
gráfokat egy .cypher típusú fájlba.  
  

\section{Blokkdiagram}

\section{A tervezett alkalmazás felépítése}